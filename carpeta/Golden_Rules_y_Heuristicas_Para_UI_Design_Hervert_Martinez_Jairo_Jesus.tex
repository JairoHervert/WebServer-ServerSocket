\documentclass[10pt]{osa-supplemental-document}
\usepackage{xcolor}
\usepackage[utf8]{inputenc}
\usepackage[spanish]{babel}
\usepackage{makeidx}
\usepackage{apacite}
\bibliographystyle{apacite}


\title{Golden Rules de Shneiderman y Heuristicas de Nielsen y Norman para el diseño de interfaces}
\author{Hervert Martínez Jairo Jesús}

\begin{document}

\begin{titlepage}
\centering
\vspace*{1cm}

\Huge  % Tamaño de fuente para el título
\textbf{Instituto Politécnico Nacional}

\vspace{0.5cm}
\LARGE
Escuela Superior de Cómputo
\vspace{1.5cm}

\LARGE
Análisis y Diseño de Sistemas
\vspace{1.5cm}

\LARGE
Grupo: 5CM3
\vspace{1.5cm}

\huge\textbf{Investigación: Golden Rules de Shneiderman y Heuristics de Nielsen Norman}
\vspace{1.5cm}

\LARGE
Alumno: Hervert Martínez Jairo Jesús
\vspace{1.5cm}

Docente: M. Maldonado Castillo Idalia

\vfill

\LARGE
\textit{Fecha de Entrega: 09 de Octubre del 2024}

\end{titlepage}

{
  \hypersetup{linkcolor=black}
  \tableofcontents
}

\maketitle

\section{Introducción}
En el campo del diseño de interfaces de usuario (UI) y la experiencia de usuario (UX), es fundamental comprender los principios que garantizan que los sistemas interactivos sean eficientes, fáciles de usar y accesibles para los usuarios. Dos marcos teóricos que destacan en este ámbito son las "Golden Rules" de Ben Shneiderman y las "Heurísticas de Usabilidad" de Jakob Nielsen y Don Norman. Ambos conjuntos de principios proporcionan directrices clave para la creación de interfaces de usuario que no solo funcionen correctamente, sino que también mejoren la interacción entre el usuario y el sistema.

\section{Objetivo}
El objetivo de esta tarea es investigar y analizar las "Golden Rules" propuestas por Ben Shneiderman y las "Heurísticas de Usabilidad" de Jakob Nielsen y Don Norman, entendiendo cómo estos principios contribuyen al diseño de interfaces efectivas y orientadas al usuario. A través de esta revisión, se busca establecer la importancia de aplicar estos conceptos para mejorar la usabilidad y la experiencia general de los usuarios al interactuar con sistemas tecnológicos.

\section{Golden Rules de Ben Shneiderman}
Ben Shneiderman es una figura clave en el campo de la interacción humano-computadora (HCI), y sus "Golden Rules" son una serie de ocho principios que guían el diseño de interfaces de usuario eficientes:
\begin{itemize}
    \item \textbf{Esforzarse por la consistencia:} Las interfaces deben mantener consistencia en comandos, terminología y acciones, permitiendo a los usuarios aprender más rápidamente y evitando confusiones.
    \item \textbf{Permitir atajos para usuarios frecuentes:} Proporcionar atajos o métodos rápidos para que los usuarios más experimentados realicen tareas repetitivas de manera más eficiente.
    \item \textbf{Ofrecer retroalimentación informativa:} Cada acción del usuario debe tener una respuesta visible del sistema que confirme que la acción fue ejecutada o que hay un problema.
    \item \textbf{Diseñar diálogos que generen un cierre:} Las secuencias de acciones deben tener un principio, desarrollo y cierre claros, proporcionando a los usuarios una sensación de progreso y logro.
    \item \textbf{Ofrecer prevención y manejo de errores:} El diseño debe prever posibles errores de los usuarios y permitir correcciones fáciles cuando ocurran.
    \item \textbf{Permitir deshacer acciones:} Los usuarios deben tener la posibilidad de deshacer y rehacer acciones, lo cual reduce la ansiedad y promueve la exploración.
    \item \textbf{Apoyar el control del usuario:} El sistema debe dar control al usuario, permitiendo que las interacciones sean predecibles y controlables.
    \item \textbf{Reducir la carga de memoria a corto plazo:} Minimizar la cantidad de información que los usuarios deben recordar en cada momento, presentando la información necesaria en el momento adecuado.
\end{itemize}

\section{Heurísticas de Usabilidad de Jakob Nielsen}
Jakob Nielsen, junto con Don Norman, desarrolló un conjunto de 10 heurísticas que sirven como pautas para evaluar y mejorar la usabilidad de un sistema interactivo:
\begin{itemize}
    \item \textbf{Visibilidad del estado del sistema:} El sistema debe mantener a los usuarios informados sobre su estado mediante retroalimentación oportuna.

    \item \textbf{Correspondencia entre el sistema y el mundo real:} El sistema debe usar un lenguaje que los usuarios comprendan fácilmente, utilizando metáforas y convenciones del mundo real.
    
    \item \textbf{Control y libertad del usuario:} Los usuarios deben poder deshacer y rehacer sus acciones fácilmente, permitiendo la recuperación rápida de errores.
    
    \item \textbf{Consistencia y estándares:} Los usuarios no deben dudar si diferentes palabras o acciones significan lo mismo; el sistema debe seguir estándares y convenciones.
    
    \item \textbf{Prevención de errores:} El diseño debe prever y evitar posibles errores, y en caso de ocurrir, proporcionar soluciones claras.
    
    \item \textbf{Reconocer antes que recordar:} Minimizar la carga de memoria de los usuarios mostrando la información relevante en lugar de depender de la memoria a corto plazo.
    
    \item \textbf{Flexibilidad y eficiencia de uso:} El sistema debe ser flexible para adaptarse tanto a usuarios principiantes como avanzados, proporcionando atajos o personalización.
    
    \item \textbf{Estética y diseño minimalista:} El diseño debe evitar información irrelevante o superflua para reducir la carga cognitiva.
    
    \item \textbf{Ayudar a los usuarios a reconocer, diagnosticar y recuperarse de errores:} Los mensajes de error deben ser claros, evitando terminología técnica y ofreciendo soluciones.
    
    \item \textbf{Ayuda y documentación:} Aunque el sistema debería ser usable sin necesidad de ayuda, debe ofrecer documentación clara cuando se necesite.
\end{itemize}

\section{Conclusión}
Tanto las "Golden Rules" de Shneiderman como las "Heurísticas de Usabilidad" de Nielsen y Norman son principios fundamentales para el diseño de interfaces de usuario que faciliten una interacción eficiente, intuitiva y amigable. Ambas teorías coinciden en que la consistencia, la retroalimentación y la prevención de errores son esenciales para mejorar la experiencia del usuario. La aplicación de estos principios en el diseño de interfaces no solo reduce la carga cognitiva del usuario, sino que también aumenta su satisfacción y productividad. En resumen, estos principios proporcionan una base sólida para crear sistemas que no solo sean funcionales, sino que también promuevan una experiencia de usuario fluida y sin frustraciones.


\section{Referencias}
\begin{itemize}
    \item Rajanen, M. (2022, August). Universal golden rule for human-technology interaction design. In Proceedings of the 8th International Workshop on Socio-Technical Perspective in IS Development (STPIS 2022). RWTH Aachen University.
    \item Sridevi, S. (2014). User interface design. International Journal of Computer Science and Information Technology Research, 2(2), 415-426.
    \item Taft, T., Staes, C., Slager, S., & Weir, C. (2016). Adapting Nielsen’s design heuristics to dual processing for clinical decision support. In AMIA Annual Symposium Proceedings (Vol. 2016, p. 1179). American Medical Informatics Association.
    \item Nielsen, J. (1995). Characteristics of usability problems found by heuristic evaluation. Nielsen Norman Group. Objavljeno, 1.
\end{itemize}

\end{document}